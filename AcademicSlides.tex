\documentclass[9pt, mathserif, aspectratio=43]{beamer}

\usepackage{AcademicSlides}
% \usepackage{ctex}


%% set for bibliography
\usepackage[doi=false,isbn=false,url=false,eprint=false,
maxcitenames = 1,
giveninits=true,
date=year]{biblatex}
\bibliography{ref}
% Hide the title of the references
\AtEveryCitekey{\clearfield{title}}

%% Set the logo at the first page
\titlegraphic{
    % \includegraphics[height=1.5cm]{fig/NJU_logo.jpg}
    % \\[0.3cm]
    % \hspace{1cm}
    % \includegraphics[height=1.5cm]{fig/UTS_logo.jpg}
    % \qquad
    % PSU logo
    % \includegraphics[height=1.5cm]{fig/PSU_logo.png}
    % SIMBA lab logo
    \vspace{-.7cm}\includegraphics[height=2.7cm]{fig/SIMBA_log_20230502103507.png}
}


\author[\textsc{Jiaxin Zhong}]{\textsc{\underline{Jiaxin Zhong}} }
\institute[PSU]{\large The Sound Innovation of Metamaterials and Biomedical Acoustics (SIMBA)\\ The Pennsylvania State University (PSU)}
\date{\today}
\title[Short Title]{{\huge Main Title} \\[0.3cm] Here Goes the Subtitle}

\begin{document}

\maketitle
\begin{frame}{Outline}
    \begin{itemize}
        \LARGE 
    \item Introduction
    \item Main section 1
    \item Main section 2
    \item Conclusions and future work
    \end{itemize}
\end{frame}

\begin{frame}{Introduction}
    This is a template for the academic presentation.
    \begin{block}{Block title}
        \begin{itemize}
            \item Line 1
            \item Line 2
        \end{itemize}
    \end{block}
    \begin{exampleblock}{Example block title}
        Description here
        \begin{itemize}
            \item Line 1
            \item Line 2
        \end{itemize}
    \end{exampleblock}
\end{frame}

\begin{frame}
    \frametitle{Sample slide}
    Sound fields on \emph{front side}:
    \begin{itemize}
        \item \emph{Near field}: second-order nonlinear or Kuznetsov equation (local effects are strong)
    \item \emph{Westervelt far field}: Westervelt equation (local effects are negligible)
    \item \emph{Inverse-law far field}: $p \propto 1/r$
            % \2 $R_1$: \emph{transition distance} from near field to Westervelt far field (0.1 m)
            % \2 $R_2$: \emph{transition distance} from Westervelt far field to inverse-law far field (30 m)
        % \1 Back side
        \item Proposed simple formulae for the \emph{transition distances}
        \end{itemize}
    \begin{columns}
        \column{0.4\textwidth}
        \begin{figure}[!htb]
        \centering
        \includegraphics[width = \textwidth]{example-image-a}
        \caption{Captions A}
    \end{figure}
    \column{0.4\textwidth}
    \begin{figure}[!htb]
        \centering
        \includegraphics[width = \textwidth]{example-image-b}
        \caption{Captions B}
    \end{figure}
    \end{columns}
    \blfootnote{
        \vspace{-0.3cm}
        \begin{enumerate}
            \item \fullcite{Wen1988DiffractionBeamField}
        \end{enumerate}
        }
\end{frame}

\begin{frame}
    \frametitle{Sample slide}
    Sound fields on \emph{back side}:
    \begin{itemize}
        \item Proposed a \emph{non-paraxial theoretical model} validated by experiments
    \item Audible sound behind a PAL especially at \emph{low frequencies} 
    \end{itemize}
\begin{figure}[!htb]
    \centering
    \begin{columns}
        \column{0.35\textwidth}
    \begin{subfigure}{\textwidth}
        \centering
        \includegraphics[width = \textwidth]{example-image-16x10}
    \end{subfigure}
    \\
    \begin{subfigure}{\textwidth}
        \centering
        \includegraphics[width = \textwidth]{example-image-16x10}
    \end{subfigure}

    \column{.35\textwidth}
    \begin{subfigure}{\textwidth}
        \centering
        \includegraphics[width = \textwidth]{example-image-16x10}
    \end{subfigure}
    \\
    \begin{subfigure}{\textwidth}
        \centering
        \includegraphics[width = \textwidth]{example-image-16x10}
    \end{subfigure}
    \column{.2\textwidth}
    \caption{Example of 2x2 images} 
    \end{columns}
\end{figure}
    \blfootnote{
        References:
        \begin{itemize}
            \item \fullcite{Wen1988DiffractionBeamField}
        \end{itemize}
        }
\end{frame}

\begin{frame}{Sample slide --- Equations}
    \emph{Using equations} 
    \begin{itemize}
        \item Equation template
\end{itemize}
    \begin{equation}
        \text{Existing method:}
        \qquad p(\vb{r}) = \iiiiint 
        \cdots
        \dd^2 \vb{r}' \dd^3 \vb{r}''
    \end{equation}
    \begin{itemize}
        \item Shading part of the equation
    \end{itemize}
    \begin{equation}
        \text{Proposed method:}
        \qquad p(\vb{r}) = \sum\sum\sum\sum 
        \tikz[baseline]{
            \node[fill=red!20, anchor=base]{
            $\displaystyle \int  \cdots \dd r'$}
        }
    \end{equation}
\end{frame}

\begin{frame}{Thanks}
    \centering 
    \Huge \emph{Thank you!\\ Any Questions?}
\end{frame}

\end{document}
